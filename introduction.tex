\chapter{Introduction} % (fold)
\label{cha:introduction}
Evolution has been, since the time of Darwin, a science mostly concerned with the past. The principles of evolution have been used to explain observations of ancient creatures and to predict what new evidence from the past should eventually turn up. One example of this sort of prediction is the existence of transitional forms. Evolution through the gradual acquisition of altered traits would predict the existence of ancient animals with a blending of traits from closely related branches in the tree of life. Indeed many such forms have been discovered, validating this prediction.

When it comes to predicting the past, evolution has been wildly successful. Where the study of Evolution has been thus far lacking is in its ability to predict the future. That is, looking at collected fossils and current natural evidence, the Theory of Evolution gives us the ability to identify which selective pressures acted on past populations. What the Theory of Evolution cannot do, at present, is predict which environmental or other influences will act as selective pressures going forward. At best, we can make educated guesses based on past evidence, but we lack even the basic ability to assign a concrete measure of confidence in such predictions.

This situation is akin to a meteorologist being able to explain why it rained yesterday but completely unable to predict tomorrow's weather. It is important, though, to understand that this is not an inherent failing of evolution, but rather a sign of a young science with much work left to do and many discoveries yet to be made. In physics, Newton's laws were successful at explaining the action of an apple falling from a tree for many centuries before they were developed to the point they could inform us how to send a man to the moon. What's remarkable about this development, from falling apples to moon landings, is that it did not require any fundamental additions to Newton's laws. (Certainly General Relativity has fundamentally altered Newton's laws, but it is not, strictly speaking, an alteration required to get to the moon.) Rather, physics was able to make this progress merely through improved tools, improved instruments, and most importantly, improved means of applying the fundamental laws to a problem.

Thus, in order to advance the field of evolution, we should strive for more and better data, but also new and better ways of analyzing and testing that data. Before looking at how this might be achieved, let us look at what the implications of better predictive power in evolution might be. A rather straight forward implication would be the ability to predict the occurrence and course of epidemic or even pandemic diseases. Such diseases are biological organisms subject to Darwinian evolution and, often times, humans are the niche for which they are adapting. We very frequently alter their niche, introducing new selective pressures, in the ways that we treat disease with medicines, quarantine, or myriad other techniques.

The implications can be more far reaching then they might first seem. Partially, this is due to the fact that evolution, and the principles of Darwinian Evolution, apply to a much more diverse range of situations than just the origin of animal and plant species. For example, cancer is an evolutionary process. With each round of chemotherapy or radiation treatment, those cancer cells that have adaptations that increase their resistance to treatment will be more likely to survive. These cells will, thus, seed what will almost inevitably form as a reemergent, more difficult to treat, tumor. Therefore, understanding the dynamics of evolution and how to predict the future course of evolution would improve our ability to design effective treatments for cancer.

Even non-biological processes are driven by evolution and obey many of the same laws that Darwin first laid out 150 years ago. Both languages and economies undergo evolution, driven by the same math as cancer or the origin of species. At this point it is useful to make a distinction between biological and non-biological evolution. The reason for doing so lies in the approaches that can be taken to investigate each type of evolutionary system. We know a great deal more about biological organisms than that they merely evolve. The past 50 years has resulted in an explosion of understanding of the chemistry of life and the operation of the molecular systems which compose cells. On the other hand, systems such as language and economy have, as their atomic components, humans and the human mind. While we understand much about the human mind, our grasp of its elementary functioning still pales in comparison to recent advances in biochemistry and molecular biology.

For this reason, while non-biological systems can be studied using an outside-in approach just as easily as biological systems, biological evolving systems present to us a unique opportunity to attempt to understand the mechanisms of evolution from the inside. Specifically, with biology we can explore the internal feedback mechanism which drives evolution: the ``Central Dogma'' of biology. This is the pathway by which information flows from an organisms nucleic acids, where it is stored, to the organisms proteins, where the information drives fundamental biochemical processes. These biochemical processes are what is eventually selected for in the process of natural selection, determining what information gets propagated, but these biochemical processes are also what carries out the propagation of that information.

To understand why it is so difficult to make predictions about the future dynamics of evolutionary systems, it helps to the details of how such systems works. Darwin's essential observations can be summed up in two important concepts: reproductive success and descent with modification. The concept of reproductive success states that those organisms which reproduce in the greatest number and at the greatest rate will have their genes increase in frequency in a population over time. Since it is these genes which determine, to some extent or another, how successful organisms will be at reproducing, reproductive success serves as a filter for so-called ``best fit'' genes. Such a filter would, ultimately, result in a highly refined population of only those organisms which represent the best possible reproducers. However, such a system is static, and we know from observation that evolving populations are dynamic.

The dynamism of evolving populations comes about from descent with modification. In every biological organism, and indeed in every reproductive unit which experiences Darwinian evolution, the process of reproduction is not carried out with perfect fidelity. Rather, in each generation new variants arise. These variants can be generated through a number of mechanisms such as the recombination of distinct genetic elements or the blending of traits from multiple parents. If we are to understand biological evolution at its most basic, however, the most important mechanism for the introduction of variation in early organisms is mutation: errors in the fundamental process of duplicating genetic information.

Restating the problem of making predictions about evolutionary processes, then, what we must be concerned with are which filters will operate on a population through the process of reproductive success and how the processes that introduce variation will be affected by and at the same time effect these reproductive success filters. This conceptual model is familiar from the science of thermodynamics. In thermodynamics a system particles is subject to certain forces. The particles are also imparted with certain momenta. The trajectories described by these momenta determine which forces and with what magnitude the particles will experience as time progresses. At the same time, the forces each particle is subject to will alter the particle's trajectory. So, in a thermodynamics inspired model of evolution, we can think of the selective filters of reproductive success as forces and the introduction of new variants as momenta. In this work I will explore the applicability of this conceptual model by looking at the specific case of nucleotide polymerases. This exploration takes the form of a model of polymerase evolution.

\section*{Nucleic Acid Polymerization} % (fold)
\label{sec:nucleotide_polymerization}
The two classes of biologically important nucleic acids are Ribonucleic Acid (RNA) and Deoxyribonucleic Acid. These two classes of molecules are very similar, differing only by the presence or absence, respectively, of a 2' hydroxyl group on the ribose sugar of their individual monomers. Both are formed primarily by a process of dehydration synthesis catalyzed by a class of enzymes known as nucleotide polymerases. The dehydration reaction takes place between one of the hydroxyl groups on the alpha phosphate of a nucleotide triphosphate and the 3' hydroxyl group of the terminal monomer on the growing nucleic acid chain.

What is phenomenal about this process is that, first, it occurs in all biological organisms and, second, that it always occurs with the same directionality. To understand why the consistency of directionality is notable, it helps to understand the catalytic process that occurs at the active center of nucleotide polymerases. All known nucleotide polymerases share a common chemical mechanism. In this mechanism, two divalent metal cations, coordinated by a number of acidic amino acids, facilitate the transfer of an electron pair from the free 3' hydroxyl group of one nucleotide to the alpha phosphate, which is attached to the 5' hydroxyl of the other nucleotide.

In this mechanism, the catalytically active cations and acidic amino acids are unbiased as to which nucleotide presents the free 3' hydroxyl and which presents the 5'-linked phosphate group. In other words, if one were to imagine a nucleotide polymerase that proceeded in the reverse direction of all currently known nucleotide polymerases, the only aspect of the naturally occurring enzymes that would need to be modified are those portions which attach to the growing nucleotide polymer or the incoming nucleotide, all of which are distinct from the catalytic center. Yet, the fact that all biological organisms polymerize both DNA and RNA with the same directionality implies that this aspect of nucleotide polymerization was decided extremely early on in the genesis of life on earth, most likely even before the biological distinction between DNA and RNA functionality arose.

If the active site mechanism of nucleotide polymerases cannot explain the apparent bias of all life toward one polymerization directionality over the other, then what alternative explanations might exist? Two possibilities are that the unique directionality is the consequence of a founder effect or that there is an evolutionary advantage to the selected directionality. A founder effect is the result of when a small subset of a larger population is evolutionarily isolated from the original population and goes on to give rise to a new population. This new population would be expected to contain an oversampling of the genetics of the founding population as compared to the gene frequencies of the original population. In the case of nucleotide polymerization, this would imply that at some point early on in the development of life on earth the population contained both forms of nucleotide polymerases, those that polymerize by extension $3'\to5'$ and $5'\to3'$ (as all life today). Then, at some later point, a subgroup of this population that contained only $5'\to3'$ polymerases was isolated and subsequently gave rise to all life on earth today.

Unfortunately, the chances of finding any evidence directly supporting the hypothesis of a founder effect determining polymerase directionality are essentially nil. Short of finding some remnant modern population with reversed polymerases, the only evidence of an ancient $3'\to5'$ polymerase would be the enzymes or other early replicator molecules themselves, and individual molecules do not fossilize. Therefore, to shed more light on which of these two competing hypotheses explains the current state of nucleotide polymerases in biology, it will be necessary to identify what possible evolutionary advantage could have been imparted on an organism by having a $5'\to3'$ polymerase instead of the reverse. In order to do this, the forces and momenta that would influence the evolution of a nucleotide polymerase in the simplest of replicating proto-organisms should be considered, in keeping with a thermodynamic approach to evolution.

Because nucleotide polymerases are responsible for replicating the genetic information of an organism, and therefore have a direct influence on both the rate at which an organism can reproduce as well as the fidelity with which genetic information is conveyed from one generation to the next, this analysis is relatively straight forward. The model presented here will involve simplified model organisms designed to represent the earliest forms of self-replicating life. For the model, it is assumed these organisms must have arisen in an environment rich with nutrients and an excess of available energy. This means that the rate of genome duplication would serve as the limiting factor on reproductive rate. This is one force that will act on the evolution of the model organisms.

In addition to reproductive rate, we must also take into account the possibility of a lower acceptable limit on genetic transmission fidelity. At the limit, a polymerase which has zero fidelity in reproducing genetic information cannot really be said to be evolving so much as randomly assembling chemicals. That is, while the dynamic nature of evolution requires the introduction of some information entropy during reproduction, without at least some of the original message being persisted across generations selective pressures cannot operate. The precise value of this lower limit will depend on a number of factors. For the model constructed here, a more pragmatic approach has been adopted. It has been observed empirically that, averaged over a large sampling of proteins and organisms, at least 76\% fidelity is required for continued function. Thus, the second force which will operate on the model organisms is that of minimum acceptable polymerase fidelity.

As for the momentum which will guide the model organisms through the evolutionary landscape, only mutation rate will be considered. Since polymerase directionality would have been fixed extremely early on in the origin of life, it is reasonable to assume that the recombination of genetic elements would contribute only a negligible amount of variation to the model organisms. Additionally, the simplest of nucleotide polymerases observed in nature are the products of single genes. Even if there was recombination of individual genetic elements, the exclusive focus of this model is the polymerase. Whether a product polymerase gene is transferred intact to the offspring of one organism or transferred horizontally will not affect the conclusions that can be made. This is a result of the first assumption made that the model organisms have an excess of all resources aside from the polymerase. That is, whatever organism a polymerase may end up in, it will always be the single determining factor for both reproductive rate and fidelity.
% section nucleotide_polymerization (end)
% chapter introduction (end)