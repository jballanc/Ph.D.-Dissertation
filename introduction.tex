\chapter{Introduction} % (fold)
\label{cha:introduction}
Evolution has been, since the time of Darwin, a science mostly concerned with the past. The principles of evolution have been used to explain observations of ancient creatures and to predict what new evidence from the past should eventually turn up. One example of this sort of prediction is the existence of transitional forms. Evolution through the gradual acquisition of altered traits would predict the existence of ancient animals with a blending of traits from closely related branches in the tree of life. Indeed many such forms have been discovered, validating this prediction.

When it comes to predicting the past, evolution has been wildly successful. Where the study of Evolution has been thus far lacking is in its ability to predict the future. That is, looking at collected fossils and current natural evidence, the Theory of Evolution gives us the ability to identify which selective pressures acted on past populations. What the Theory of Evolution cannot do, at present, is predict which environmental or other influences will act as selective pressures going forward. At best, we can make educated guesses based on past evidence, but we lack even the basic ability to assign a concrete measure of confidence in such predictions.

This situation is akin to a meteorologist being able to explain why it rained yesterday but completely unable to predict tomorrow's weather. It is important, though, to understand that this is not an inherent failing of evolution, but rather a sign of a young science with much work left to do and many discoveries yet to be made. In physics, Newton's laws were successful at explaining the action of an apple falling from a tree for many centuries before they were developed to the point they could inform us how to send a man to the moon. What's remarkable about this development, from falling apples to moon landings, is that it did not require any fundamental additions to Newton's laws. (Certainly General Relativity has fundamentally altered Newton's laws, but it is not, strictly speaking, an alteration required to get to the moon.) Rather, physics was able to make this progress merely through improved tools, improved instruments, and most importantly, improved means of applying the fundamental laws to a problem.

Thus, in order to advance the field of evolution, we should strive for more and better data, but also new and better ways of analyzing and testing that data. Before looking at how this might be achieved, let us look at what the implications of better predictive power in evolution might be. A rather straight forward implication would be the ability to predict the occurrence and course of epidemic or even pandemic diseases. Such diseases are biological organisms subject to Darwinian evolution and, often times, humans are the niche for which they are adapting. We very frequently alter their niche, introducing new selective pressures, in the ways that we treat disease with medicines, quarantine, or myriad other techniques.

The implications can be more far reaching then they might first seem. Partially, this is due to the fact that evolution, and the principles of Darwinian Evolution, apply to a much more diverse range of situations than just the origin of animal and plant species. For example, cancer is an evolutionary process. With each round of chemotherapy or radiation treatment, those cancer cells that have adaptations that increase their resistance to treatment will be more likely to survive. These cells will, thus, seed what will almost inevitably form as a reemergent, more difficult to treat, tumor. Therefore, understanding the dynamics of evolution and how to predict the future course of evolution would improve our ability to design effective treatments for cancer.

Even non-biological processes are driven by evolution and obey many of the same laws that Darwin first laid out 150 years ago. Both languages and economies undergo evolution, driven by the same math as cancer or the origin of species. At this point it is useful to make a distinction between biological and non-biological evolution. The reason for doing so lies in the approaches that can be taken to investigate each type of evolutionary system. We know a great deal more about biological organisms than that they merely evolve. The past 50 years has resulted in an explosion of understanding of the chemistry of life and the operation of the molecular systems which compose cells. On the other hand, systems such as language and economy have, as their atomic components, humans and the human mind. While we understand much about the human mind, our grasp of its elementary functioning still pales in comparison to recent advances in biochemistry and molecular biology.

For this reason, while non-biological systems can be studied using an outside-in approach just as easily as biological systems, biological evolving systems present to us a unique opportunity to attempt to understand the mechanisms of evolution from the inside. Specifically, with biology we can explore the internal feedback mechanism which drives evolution: the ``Central Dogma'' of biology. This is the pathway by which information flows from an organisms nucleic acids, where it is stored, to the organisms proteins, where the information drives fundamental biochemical processes. These biochemical processes are what is eventually selected for in the process of natural selection, determining what information gets propagated, but these biochemical processes are also what does the propagation of that information.

To understand why it is so difficult to make predictions about the future dynamics of evolutionary systems, it helps to the details of how such systems works. Darwin's essential observations can be summed up in two important concepts: reproductive success and descent with modification.
% chapter introduction (end)