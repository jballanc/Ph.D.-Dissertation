\appendix
\chapter{Source Code} % (fold)
\label{cha:source_code}

\lstset{language=ruby, numbers=left, basicstyle=\scriptsize \tt}
\begin{lstlisting}
# Copyright (c) 2009 Joshua Ballanco
#
# class Environment
#
# Abstract: The Environment class "contains" the entire simulation. It is
# primarily responsible for tracking the component organisms, removing dead
# organisms, and stepping each organism at each time step of the simulation.

# The GenomeForEnvironment class is used to seed the starting population of
# the environment. It contains a +Genome+ object and the frequency at which
# that genome should exist in the starting population.
GenomeForSpecies = Struct.new(:genome, :population_frequency)

class Environment
  attr_reader :temperature

  # The +Environment+ must be initialized with the size of the initial
	# population, the temperature of the simulation (in units of h-bond energy),
	# and an array of structs describing the genomes to be use in creating the
	# initial organisms.
  def initialize(temperature, max_population, starting_population, *genomes_for_environment)
    
		# The population frequency of all the genomes must add to 1
    unless genomes_for_environment.inject(0.0) {|total, gfe|
			BigDecimal(String(total + gfe.population_frequency))
		} == 1
      raise ArgumentError, "population frequencies of genomes must total 1"
    end

    if starting_population > max_population
      raise ArgumentError, "the starting population must be less than the
														maximum population"
    end

    @temperature = temperature
    @max_population = max_population
    @organisms = []
    genomes_for_environment.each do |genome_for_species|
      (starting_population * genome_for_species.population_frequency).round.times do
        @organisms << Organism.new(genome_for_species.genome.dup, self)
      end
    end
  end

  # Runs the environment for _max_iterations_ rounds (default is 1000).
  def run(iterations=1000)
    iterations.times { step }
  end

  # Set the number of threads to use, if we want to run the simulation
	# threaded.
  def use_threads(num_threads)
    @num_threads = num_threads
  end

  # Before stepping the environment, calculate the probability that any
	# individual organism will die due to resource constraints. This is modeled
	# as a agregate probability of $\frac{1}{(N-n)+1}$, evenly distributed over
	# the organisms in the environment, where $N$ is the carrying capacity of
	# the environment (_max_population_) and $n$ is the number of organisms
	# currently in the environment. At each step, the organism will either die
	# and return nil or step and return self. At the end of stepping each
	# organism, we compact the array to remove dead organisms.
  def step
    if (@num_threads && @num_threads > 1)
      @organisms.threadify(@num_threads) do |organism|
        organism.step
      end
    else
      @organisms.each do |organism|
        organism.step
      end
    end

    # This is the probability that 1 organism will die:
    death_expect = 1.0 / ((@max_population - @organisms.length) + 1)
    if rand < death_expect
      @organisms.delete_at(rand(@organisms.length))
    end
  end

  # The _add_organism_ method attempts to add an organism to the environment. If there adequate capacity, the
  # organism is added and the method returns +true+. If the environment is currently full, then nothing is done
  # and the method returns +false+.
  def add_organism(organism)
    if @organisms.length < @max_population
      @organisms << organism
      return true
    else
      return false
    end
  end

  # The _report_ method returns a hash containing the values for this environment as well as the results of
  # iterating over the @organisms array and calling each organism's _report_ method in turn.
  def report
    { :temperature => @temperature,
      :max_population => @max_population,
      :current_population => @organisms.length,
      :organisms => @organisms.collect{|organism| organism.report} }
  end
end
\end{lstlisting}
% chapter source_code (end)