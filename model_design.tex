\chapter{Design of the Polymerase Evolution Model} % (fold)
\label{cha:design_of_the_polymerase_evolution_model}
This chapter provides a detailed description of the design of the model system used to investigate the question of the evolution of nucleotide polymerase directionality. The goal of this model is to have organisms containing either a $5'\to3'$ or a $3'\to5'$ nucleic acid polymerase compete with each other in order to see which of the strategies, if either, are evolutionarily dominant. In the model, evolutionary pressure is applied by the rate at which organisms can reproduce and the fidelity with which they duplicate their genomes. Evolutionary momentum is introduced through simple genetic mutation. The model is also designed such that the influence of temperature on the outcome of this competition can be investigated.

\section*{Overview} % (fold)
\label{sec:overview}
In simplest terms, this model consists of a number of individual organisms in competition with each other. The model can be largely divided up into, and thought about most clearly as, four interacting pieces. These are the environment, the organisms, the genomes, and the polymerases. The relationship between these pieces begins with the environment. Each run of the simulation involves exactly one environment. An environment may contain many organisms, up to a predefined carrying capacity, and at the beginning of a simulation run the environment is pre-populated organisms according to a set rules.

Each organism contains one genome. At the start of an organisms ``life'', it uses its genome to create a polymerase. The polymerase replicates either $5'\to3'$ or $3'\to5'$ and may not change direction. The polymerase then replicates the genome at a predetermined rate and with a random chance of introducing errors (mutations). When the polymerase is finished duplicating the genome, the organism will attempt to divide by binary fission using the newly created genome. If it is successful in doing so, then the new organism is added to the environment and begins a fresh life cycle while the original organism resets itself and also begins a fresh cycle.

A number of generalizations and assumptions have been made in order to render the problem being investigated tractable, but a consistent effort has been made to remain as true to life as possible. The motivation behind this is that, while the exact values generated by this model may not be precisely those that would be observed in the laboratory, the trends observed should hold true when transferred to the bench. For each piece decisions have been made regarding which aspects of the component are explored in depth, and which aspects are neglected, with an eye to the larger goal of investigating the role of polymerase directionality. Following is a detailed look at each component in turn.
% section overview (end)

\section*{Environment} % (fold)
\label{sec:environment}
At the start of a simulation run, the environment for that run is initialized with a starting population of organisms. To create this starting population, a set of organisms with their genome length, polymerase rate, and polymerase directionality specified is divided according each organism's designated frequency in the starting population. The starting population size can be any number up to the specified maximum population of the environment.

A key defining feature of this model is that the environment is constrained in some ways, but not in others. Specifically, the number of organisms that can simultaneously co-exist has a hard numerical limit; a carrying capacity. On the other hand, the amount of available energy, the quantity of activated nucleotide triphosphates, and the other raw materials required for forming a cell are all considered unlimited. In reality, the carrying capacity is a generalization that could correspond to physical space limits, but it could just as easily correspond to the aggregate limitations on the other resources not explicitly modeled.

The choice of limitation based on carrying capacity was driven by two factors. The first relates to the fact that selection during an exponential growth phase could potentially operate differently than during stationary phase growth. By placing an upper limit on the size of a population, it is possible to investigate selection during both growth phases. Second, while attempting to explicitly model all of the various resource and space constraints that might ultimately limit growth might yield a more complete model, it would also significantly increase the complexity of the model without adding much insight into the question at hand. Finally, numerically or density limited growth is a common observation across a wide range of living creatures from single cells to large populations of complex animals.

In order to model density dependent growth inhibition as the number of organisms approachs the carrying capacity, a random death probability is introduced to the environment. In keeping with observations that the pressure of density dependant inhibition is greater as the population of an environment approaches the carrying capacity, the death probability is calculated with an inverse function of the remaining capacity\[
	P_{death} = \frac{1}{(N-n)+1}
\]where $N$ is the carrying capacity, $n$ is the number of organisms currently in the environment, and $1$ is added so that the probability of at least one organism being culled from the environment when the carrying capacity is reached is $P_{death}=1$.

The model iterates its environment in a stepwise fashion. During each step, each of the organisms contained within the environment is allowed to carry out one time-step of its life cycle, followed by a population culling. Culling is carried out by calculating the death probability as above, and then randomly applying that probability to the environment to decide if an organism should be removed. If the decision is made to remove an organism, one is chosen from the environment randomly, removed, and the death probability is recalculated and reapplied. This process will repeat until the decision is made not to remove an organism. In this way, the number of organisms removed at each time step follows a Poisson distribution with an expectation value of $P_{death}$.
% section environment (end)

\section*{Organism} % (fold)
\label{sec:organism}
Organisms are created and added to the environment either at the start of a simulation run, as part of the starting population, or during the simulation run as the result of binary fission of an existing organism. If the organism is created at the start of the simulation then its genome and polymerase properties are determined by the values used to seed the population. If it is created during a simulation run, then its properties are derived from those of its parent, with the possibility of introduced variation. This variation is embodied by, and in the model determined by, the genome contained within the organism. Each organism contains exactly one genome and one polymerase. Having only one polymerase is a simplification, but it is justified by considering this one polymerase as an exemplar of the various polymerase enzymes that would be found in a real organism.

Each organism is modeled as a state machine. The two states in which they can exist are: \emph{Polymerizing} or \emph{Duplicating}. At creation, each organism starts in the \emph{Polymerizing} state. When an organism is in the \emph{Polymerizing} state, each simulation time-step is used to allow the organism's polymerase to add nucleotides to the genome copy being constructed. At the end of each time-step, the polymerase is queried as to whether or not it is finished with constructing the nascent genome. If it is, then the organism shifts to the \emph{Duplicating} state. Otherwise, it remains in the \emph{Polymerizing} state.

When an organism reaches the \emph{Duplicating} state, the first task is to determine if the genome copy constructed by the polymerase is viable. This determination is made by comparing the fraction of errors made by the polymerase during polymerization against the empirically observed upper tolerance of 76\%. If the genome copy is not viable, then it is discarded and the organism returns to the \emph{Polymerizing} state to create a new genome copy. That is, no allowance is made for damage repair. This is yet another simplifying assumption, but considering that the model aims to simulate the veriest early forms of life, it is doubtful that complex \emph{post facto} error correcting mechanisms would have existed (or that they would have contributed significantly to the evolution of such early organisms).

If the genome is viable, the next determination that must be made is whether or not there is available capacity in the environment for a new organism.
% section organism (end)

% chapter design_of_the_polymerase_evolution_model (end)