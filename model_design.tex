\chapter{Design of the Polymerase Evolution Model} % (fold)
\label{cha:design_of_the_polymerase_evolution_model}
This chapter provides a detailed description of the design of the model system used to investigate the question of the evolution of polymerase polarity. The goal of this model is to have organisms containing either a $5'\to3'$ or a $3'\to5'$ nucleic acid polymerase compete with each other in order to see which strategies are either evolutionarily dominant or evolutionarily stable. The model is also designed such that the influence of temperature on the outcome of this competition can be investigated.

\section*{Overview} % (fold)
\label{sec:overview}
In simplest terms, this toy model consists of a number of individual organisms competing with each other in an environment. A number of generalizations and assumptions have been made in order to render the problem being investigated tractible, but a consistent effort has been made to remain as true to life as possible. The motivation behind this is that, while the exact values generated by this model may not be precisly those that may be observed in the laboratory, the trends observed should hold true when transfered to the bench.

The model can be largely divided up into, and thought about most clearly as, four interacting pieces. These are the environment, the organisms, the genomes, and the polymerases. For each piece decisions have been made regarding which aspects of the component are explored in depth, and which aspects are neglected, with an eye to the larger goal of investigating the role of polymerase polarity. Following is a detailed look at each component, including justifications for each decision made in the design.
% section overview (end)

\section*{Environment} % (fold)
\label{sec:environment}
A key defining feature of the model is that the environment is constrained in some ways, but not in others. Specifically, the number of organisms that can simultaneously co-exist has a hard numerical limit; a carrying capacity. On the other hand, the amount of available energy, the quantity of activated nucleotide triphosphates, and the other raw materials required for forming a cell are all considered unlimited.

In order to mimic, at least empirically, the sort of density dependant inhibition to growth that is observed in all cells as the approach the carrying capacity of their environment, a random death probability is introduced to the environment. In keeping with observations that the pressure of density dependant inhibition is greater as the population of an environment approaches the carrying capacity, the death probability is calculated with an inverse function of the remaining capacity:
% section environment (end)

% chapter design_of_the_polymerase_evolution_model (end)