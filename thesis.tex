%
%  Thesis
%
%  Created by Joshua Ballanco on 2008-03-31.
%  Copyright (c) 2008 . All rights reserved.
%
\documentclass[]{book}

% Use utf-8 encoding for foreign characters
\usepackage[utf8]{inputenc}

% Setup for fullpage use
\usepackage{fullpage}

% Uncomment some of the following if you use the features
%
% Running Headers and footers
%\usepackage{fancyhdr}

% Multipart figures
%\usepackage{subfigure}

% More symbols
%\usepackage{amsmath}
%\usepackage{amssymb}
%\usepackage{latexsym}

% Surround parts of graphics with box
\usepackage{boxedminipage}

% Package for including code in the document
\usepackage{listings}

% If you want to generate a toc for each chapter (use with book)
\usepackage{minitoc}

% This is now the recommended way for checking for PDFLaTeX:
\usepackage{ifpdf}

%\newif\ifpdf
%\ifx\pdfoutput\undefined
%\pdffalse % we are not running PDFLaTeX
%\else
%\pdfoutput=1 % we are running PDFLaTeX
%\pdftrue
%\fi

\ifpdf
\usepackage[pdftex]{graphicx}
\else
\usepackage{graphicx}
\fi
\title{The Study of Life as an Energetic Process}
\author{Joshua Ballanco}

\date{2008-03-31}

\begin{document}

\ifpdf
\DeclareGraphicsExtensions{.pdf, .jpg, .tif}
\else
\DeclareGraphicsExtensions{.eps, .jpg}
\fi

\maketitle

\chapter{Introduction} % (fold)
\label{cha:introduction}
It is my hypothesis that life, specifically the evolution of life, is an energetic process. This hypothesis is driven by the observation that life is, at its most fundamental level, nothing more than an elaborate collection of chemical reaction. Taken individually, each of these reactions is governed by the laws of thermodynamics. Furthermore, it is well known in the field of chemistry that, in a system of multiple competing reactions, those reaction which occur most rapidly will dominate on short time scales, but those whose products are more thermodynamically stable will prevail over longer time scales. This is the concept of kinetic control vs thermodynamic control. Life, then, should also be subject to aspects of both kinetic and thermodynamic control, with one important caveat: life is a non-equilibrium process.\
Were we to consider the equilibrium state, we would find life processes to be rather untenable. As a simple example, consider that the average human turns over a mass of ATP equivalent to their body mass every 24 hours.
% chapter introduction (end)

\bibliographystyle{plain}
\bibliography{}
\end{document}
